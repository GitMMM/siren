% -*- latex -*-
%
This software may only be used by you under license from the
University of Notre Dame.  A copy of the University of Notre Dame's
Source Code Agreement is available at the inilib Internet website
having the URL: <http://inilib.sourceforge.net/license/> If you
received this software without first entering into a license with the
University of Notre Dame, you have an infringing copy of this software
and cannot use it without violating the University of Notre Dame's
intellectual property rights.
% 
% $Id: op_overload.tex,v 1.7 2000/09/04 01:20:46 bwbarrett Exp $
%

\subsection{Attribute Operator Overloading}
\label{sec:op_overload}

Due to the overloading of casting operators for the attribute classes
(allowing an attribute to be cast to one of {\tt bool}, {\tt double},
{\tt int}, or {\tt std::string}), compiler ambiguities can arise.  Two
solutions to the problem of compiler ambiguties exist.  The first is
to require the programmer to explicitly cast the attribute in any
operation that may result in a compiler ambiguity.  This option
requires more work for the programmer, which is not what a library
should do.  The second option is to overload the most common operators
for the attribute class, eliminating any compiler ambiguities that may
arise.  The second option is obviously a better solution for the
programmer, and is therefore implemented in {\tt inilib}.
Table~\ref{tbl:overloaded} provides a listing of the operators
overloaded in {\tt inilib}.

\begin{table}
  \begin{center}
    \begin{tabular}[ht!]{| l | l |} \hline
      pre increment & ++lvalue \\ \hline
      pre decrement & -{}-{}lvalue \\ \hline
      not & !expr \\ \hline
      unary minus & -expr \\ \hline
      multiply & expr * expr \\ \hline
      divide & expr / expr \\ \hline
      modulo & expr \% expr \\ \hline
      add (plus) & expr + expr \\ \hline
      subtract (minus) & expr - expr \\ \hline
      less than & expr $<$ expr \\ \hline
      less than or equal & expr $<$= expr \\ \hline
      greater than & expr $>$ expr \\ \hline
      greater than or equal & expr $>$= expr \\ \hline
      equal & expr ={}={} expr \\ \hline
      not equal & expr !={} expr \\ \hline
      multiply and assign & lvalue *= expr \\ \hline
      divide and assign & lvalue /= expr \\ \hline
      module and assign & lvalue \%= expr \\ \hline
      add and assign & lvalue += expr \\ \hline
      subtract and assign & lvalue -= expr \\ \hline
    \end{tabular}
  \end{center}
  \caption{Overloaded operators provided by {\tt inilib}.}
  \label{tbl:overloaded}
\end{table}

\subsubsection{Binary Operators}

The overloaded binary operators fall into two categories: {\tt
  attribute}/{\sc Datatype} functions and {\tt attribute}/{\tt
  attribute} functions.  The {\tt attribute}/{\sc Datatype} functions
will cast the {\tt attribute} to the same type as the other argument
and then perform the operation.

The {\tt attribute}/{\tt attribute} binary operators do not have an
implied type to cast to, as an {\tt attribute} can be cast to any one
of {\tt bool}, {\tt double}, {\tt int}, or {\tt std::string}.
Therefore, the {\tt attribute}/{\tt attribute} functions use the
following hierarchy for casting (listed from lowest to highest):

\begin{enumerate}
\item {\tt std::string}
\item {\tt bool}
\item {\tt int}
\item {\tt double}
\end{enumerate}

The {\tt attribute} with the higher type acording to the hierarchy
will be cast to the type of the other {\tt attribute}.  For example,
if the comparison operator is called on a {\tt string\_attribute} and
{\tt double\_attribute}, the casting operator will be used to cast
both attributes to type {\tt double}, then the operation will be
performed.  Although an arbitrary hierarchy, it is used because it is
similar to the one used by Perl and PHP to control casting from one
type to another.

The {\tt attribute}/{\tt attribute} binary arithmetic operators create
a temporary attribute of the appropriate type (as discussed in the
previous paragraph) and use the arithmentic operation and assign
member functions to perform the arithmetic operation.  Therefore, the
behavior of the {\tt attribute}/{\tt attribute} binary arithmetic
operators is determined by the operation of the member arithmetic
operation.

\subsubsection{Unary Operators}

The unary operators are all member functions of the {\tt attribute}
class.  Some of the unary operators (pre increment, for instance), are
not defined in C/C++ for some data types ({double}, in the case of the
pre increment operator).  However it would be inconvienent not to have
the operators defined for all four types used by {\tt inilib}.
Table~\ref{tbl:unaryops} describes the result of many unary operators
when called on an {\tt attribute}.  The action of these operators is
intended to mimic Perl and PHP as closely as possible.\footnote{The one
  exception is integer and string addition.  The incrementing of the
  alphanumerical string is not supported in {\tt inilib}}

The post-increment and post-decrement operators are not provided, as it
is not possible to return an object of type {\tt attribute}, which
would be required for the post-increment and post-decrement operators.

\begin{center}
  \begin{longtable}[!ht]{ | p{.82in} | p{.82in} | p{.82in} | p{.82in} | p{1.4in} |}
    \caption{Behavior of unary operators overloaded in {\tt inilib}.
      \protect}
    \label{tbl:unaryops} \\
    \hline
    & \multicolumn{4}{| c |}{{\tt attribute} type} \\ 
    & \multicolumn{1}{|c|}{{\tt bool}} & \multicolumn{1}{c|}{{\tt
    double}} & \multicolumn{1}{c|}{{\tt int}} &
    \multicolumn{1}{c|}{{\tt std::string}} \\ 
    \hline 
    \endfirsthead % Everything above goes at the top of the 1st page only
    \caption[]{(cont.)} \\
    \hline
    & \multicolumn{1}{|c|}{{\tt bool}} & \multicolumn{1}{c|}{{\tt
    double}} & \multicolumn{1}{c|}{{\tt int}} &
    \multicolumn{1}{c|}{{\tt std::string}} \\ 
    \hline 
    \endhead % Everything above here (and below the \endfirsthead) goes at the top
                                % of continuation pages.  The [] argument prevents a duplicate
                                % entry from appearing in the table of contents.
    \hline
    \multicolumn{5}{|r|}{\textit{continued}\ldots} \\
    \hline
    \endfoot % The above section goes at the bottom of continuation pages
    \hline
    \endlastfoot % The very last bottom of the table
    \hline
    
    pre increment & 
    sets the value of the attribute to {\tt true} and
    returns the new value & 
    adds {\tt 1} to the value of the attribute and returns the new
    value & 
    adds {\tt 1} to the value of the attribute and returns the new
    value & 
    If the value of the attibute is an int or double,\footnote{Meaning it
      contains a string of the form dddd or ddd.ddd, where d is a decimal
      digit, with an arbitrary number of digits.} the value is cast to
    that type and {\tt 1} is added to the value.  The result is converted
    back to a string.  The resulting string is assigned to the attribute
    and returned. \\ \hline
    
    pre decrement & 
    sets the value of the attribute to {\tt false} and returns the new
    value & 
    subtracts {\tt 1} from the value of the attribute and returns
    the new value & 
    subtracts {\tt 1} from the value of the attribute and returns
    the new value & 
    if the value of the attibute is an int or double, the value is
    cast to that type and {\tt 1} is subtracted from the value.  The
    result is converted back to a string.  The resulting string is
    assigned to the attribute and returned. \\ \hline
    
    not &
    returns {\tt true} if value is {\tt false}, returns {\tt false}
    if the value is {\tt true} & 
    returns {\tt true} if the value is {\tt 0}, returns {\tt false}
    otherwise & 
    returns {\tt true} if the value is {\tt 0}, {\tt false} otherwise &
    returns {\tt true} if the string is empty, {\tt false} otherwise \\
    \hline
    
    unary minus & 
    same as {\tt not} & 
    returns value, but with sign reversed & 
    returns value, but with sign reversed & 
    if the string contains only a number (integer or decimal), 
    returns the value as a string, but with a `-' prepended.  Otherwise,
    returns the current value \\ \hline
    
    multiply and assign & 
    casts argument to {\tt bool} and performs a logical {\tt and},
    returning the result and assigning it to the attribute & 
    casts argument to {\tt int} and multiplies, returning the result and
    assigning it to the attribute &
    casts argument to {\tt double} and multiplies, returning the result
    and assigning it to the attribute &
    If the argument is a {\tt bool}, {\tt double}, or {\tt int}, casts
    attribute to that type and performs the multiplication.  The result
    is converted back to a string, returned, and assigned to the
    attribute.  If the argument is a string, the {\tt std::string}'s +
    operator is used.  If the argument is an attribute, the action is as
    above, based on the underlying type. \\ \hline
    
    divide and assign & 
    casts argument to {\tt bool} and performs a logical {\tt or},
    returning the result and assigning it to the attribute &
    casts argument to {\tt int} and divides, returning the result and
    assigning it to the attribute &
    casts argument to {\tt double} and divides, returning the result and
    assigning it to the attribute &
    If the argument is a {\tt bool}, {\tt double}, or {\tt int}, casts
    attribute to that type and performs the division.  The result
    is converted back to a string, returned, and assigned to the
    attribute.  If the argument is a string, the operation returns the
    current string.  If the argument is an attribute, the action is as
    above, based on the underlying type. \\ \hline
    
    modulo and assign &
    casts argument to {\tt bool} and performs a logical {\tt and},
    returning the result and assigning it to the attribute &
    casts argument to {\tt int} and performs modulo operation, returning
    the result and assining it to the attribute. &
    casts argument and value to {\tt int} and performs modulo
    operation.  Result is cast back to a {\tt double} & 
    If the argument is a {\tt bool}, {\tt double}, or {\tt int}, casts
    attribute to that type and performs the modulo, using the rules
    specified for the {\tt bool}, {\tt double}, and {\tt int} attributes.
    The result is converted back to a string, returned, and assigned to the
    attribute.  If the argument is a string, the operation returns the
    current string.  If the argument is an attribute, the action is as
    above, based on the underlying type. \\ \hline
    
    add and assign & 
    casts argument to {\tt bool} and performs a logical {\tt
      or}, returning the reuslt and assigning it to the attribute & 
    casts argument to {\tt int} and adds, returning the result and
    assigning it to the attribute &
    casts argument to {\tt double} and adds, returning the result and
    assigning it to the attribute &
    If the argument is a {\tt bool}, {\tt double}, or {\tt int}, casts
    attribute to that type and performs the addition.  The result
    is converted back to a string, returned, and assigned to the
    attribute.  If the argument is a string, the operation returns the
    current string.  If the argument is an attribute, the action is as
    above, based on the underlying type. \\ \hline

    subtract and assign & 
    casts argument to {\tt bool} and performs a logical {\tt and},
    returing the result and assigning it to the attribute &
    casts argument to {\tt int} and subtracts, returning the result and
    assigning it to the attribute &
    casts argument to {\tt double} and subtracts, returning the result
    and assigning it to the attribute &
    If the argument is a {\tt bool}, {\tt double}, or {\tt int}, casts
    attribute to that type and performs the subtraction.  The result
    is converted back to a string, returned, and assigned to the
    attribute.  If the argument is a string, the operation returns the
    current string.  If the argument is an attribute, the action is as
    above, based on the underlying type. \\ \hline

  \end{longtable}
\end{center}
