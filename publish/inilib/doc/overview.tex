% -*- latex -*-
%
This software may only be used by you under license from the
University of Notre Dame.  A copy of the University of Notre Dame's
Source Code Agreement is available at the inilib Internet website
having the URL: <http://inilib.sourceforge.net/license/> If you
received this software without first entering into a license with the
University of Notre Dame, you have an infringing copy of this software
and cannot use it without violating the University of Notre Dame's
intellectual property rights.
% 
% $Id: overview.tex,v 1.10 2000/09/04 01:20:46 bwbarrett Exp $
%

\section[inilib Overview]{{\tt inilib} Overview}
\label{sec:overview}

The library implements a three-level hierarchy of {\tt registry}
$\rightarrow$ {\tt section} $\rightarrow$ {\tt attribute}.  A {\tt
registry} contains zero or more {\tt section}s.  A {\tt section}
contains zero or more {\tt attribute}s.  Each {\tt attribute} is a
single (key, value) pair.  To avoid name conflicts, the entire {\tt
inilib} package is in the {\tt INI} namespace.

The {\tt registry} is the top-most level of the data structure.  It
provides methods for accessing the sections it contains, as well as
reading and writing its sections from disk.  The {\tt section} class
provides access to its attributes.  Attributes are classes that
actually contain the data to be stored.  To access a particular
attribute, the programmer would use the syntax (where {\tt
section\_name} and {\tt attribute\_name} are both of type {\tt
std::string}):

\begin{verbatim}
     registry_name[section_name][attribute_name]
\end{verbatim}

Attributes can store of the following C/C++ types: {\tt
  double}, {\tt int}, {\tt std::string}, and {\tt bool}.  The
interface is designed to mimic the STL {\tt map}~\cite{STL} as closely
as possible.

The {\tt registry} class provides an interface to the sections stored
in the program.  Sections can be added by using {\tt operator[]()} or
{\tt insert()}, similar to the way a new element is added to an STL
{\tt map}.  {\tt operator[]()} can be used to retrieve a section.
Iterators are provided to traverse the sections contained in the
registry.  An STL-like {\tt find()} function is also provided to
locate a specific section.  Finally, {\tt file\_read()} and {\tt
  file\_write()} read and write all of the registry's sections.

The {\tt section} class operates in a manner similar to the {\tt
  registry} class, with the exception that it does not contain the
  file functions found in the registry class.

The attributes, which actually store data for the user, provide
interfaces for assignment (by overloading of {\tt operator=()}) from
any of the C/C++ types {\tt bool}, {\tt double}, {\tt int}, and {\tt
std::string}.  In addition, attributes can be cast to any of these
types.  The common unary and binary operators are overloaded for the
{\tt attribute} class in order to remove any possible compiler
ambiguities that might arise.  Section~\ref{sec:op_overload} provides
more information on the overloaded operators.

Figure~\ref{fig:sample_program} provides a brief example of the
general use of {\tt inilib}.  The program will create a {\tt registry}
and populate it with the default values for the sample text editor
from Section~\ref{sec:Introduction}.  The file {\tt options.ini} is
then loaded to import the user's options into the {\tt registry}.
The program then demonstrates changing an {\tt attribute}'s value.

\begin{figure}[!h]
\mybox{\#include <iostream> \\
\#include <string> \\
\#include "inilib.h" \\
\\
using namespace std; \\
\\
int \\
main(int argc, char *argv[]) \\
$\{$ \\
\hspace*{1em}INI::registry information("user.ini", false, true); \\
\\
\hspace*{1em}// Provide default values for many of the fields. \\
\hspace*{1em}information["window settings"]["width"] = 300; \\
\hspace*{1em}information["window settings"]["height"] = 300; \\
\hspace*{1em}information["history"]["file1"] = ""; \\
\hspace*{1em}information["history"]["file2"] = ""; \\
\hspace*{1em}information["history"]["file3"] = ""; \\
\\
\hspace*{1em}// Now, load the user's information \\
\hspace*{1em}information.file\_read(); \\
\\
\hspace*{1em}// And pretend the user just changed the screen width \\
\hspace*{1em}information["window settings"]["width"] = new\_width; \\
\\
\hspace*{1em}return 0; \\
\hspace*{1em}// Note that the registry is written out upon exit \\
$\}$ }
\caption{A sample program loading data from a file.}
\label{fig:sample_program}
\end{figure}

Figure~\ref{fig:sample_program} also demonstrates the connection
between a file and a registry.  The constructor associates the
filename {\tt user.ini} with {\tt information}, and instructs {\tt
inilib} not to load the information in {\tt user.ini} immediately, but
to write the {\tt registry} to {\tt user.ini} when the destructor is
called.  The topic is covered further in Section~\ref{sec:registry}.
