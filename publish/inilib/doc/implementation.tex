% -*- latex -*-
%
This software may only be used by you under license from the
University of Notre Dame.  A copy of the University of Notre Dame's
Source Code Agreement is available at the inilib Internet website
having the URL: <http://inilib.sourceforge.net/license/> If you
received this software without first entering into a license with the
University of Notre Dame, you have an infringing copy of this software
and cannot use it without violating the University of Notre Dame's
intellectual property rights.
% 
% $Id: implementation.tex,v 1.8 2000/09/04 01:20:46 bwbarrett Exp $
%

\section{Implementation}

\subsection[registry and section]{{\tt registry} and {\tt section}}

The {\tt registry} and {\tt section} classes each contain an STL {\tt
  map} with {\tt (std::string)} type for keys and values of type {\tt
  (section*)} and {\tt (attribute*)}, respectively.  The {\tt
  operator[]()}, {\tt insert()}, and {\tt clear()} functions provided
by {\tt registry} and {\tt section} perform error checking then call
the corresponding {\tt map} functions.  All functions in the two
classes, except for the file access functions {\tt file\_\-read()} and
{\tt file\_\-write()}, are inlined to increase the performance of the
library.

\subsection[attribute]{\tt attribute}
\label{sec:implementation:attribute}

The implementation of attributes is slightly more interesting than
that of the {\tt registry} and {\tt section}.  There is a base class,
{\tt attribute}, from which five other classes are derived:

\begin{enumerate}
\item {\tt bool\_\-attribute}
\item {\tt double\_\-attribute}
\item {\tt int\_\-attribute}
\item {\tt string\_\-attribute}
\end{enumerate}

The four derived classes, {\tt bool\_\-attribute}, {\tt
  double\_\-attribute}, {\tt int\_\-attribute}, and {\tt
  string\_\-attribute} (page~\pageref{sec:attribute}), are very
similar.  They provide casting operators to {\tt bool}, {\tt double},
{\tt int}, and {\tt std::string}, as well as {\tt operator=()} from these
four types.  This allows the seamless assignment and casting of any
attribute type to any other attribute type, similar to the
Perl~\cite{perl} and PHP~\cite{php} scripting languages.

The attribute class is used in situations where the type of the
attribute is not known at the time the attribute is created, for the
return type of many of the overloaded operators, and for storage in
the {\tt section} class.  The {\tt attribute} class provides the same
interface as the four classes that derive from it.  However, instead
of containing an actual data value, the {\tt attribute} class contains
a pointer to one of the other four classes.  The interaction between
the {\tt attribute} class and the derived classes is completely
transparent to the programmer.

The library is intended to be used in a situation where the programmer
provides a set of default attributes and values within the program.
When reading attributes from a file, entries from the file override
the values in memory.  However, the file value's type is assumed to be
the same as the attribute in the memory.  If an attribute exists in
the file being loaded, but is not already in memory, the parser must
make a best-guess as to the type of {\tt attribute} to create.  If the
value is whitespace, followed by an optional {\tt +} or {\tt -}, a
series decimal digits, a decimal point, then a series of decimal
digits, followed by any amount of whitespace, {\tt inilib} will
consider the value to be of type {\tt double}.  If the value is any
amount of whitespace, followed by one or more decimal digits, followed
by any amount of whitespace, {\tt inilib} will consider the value to
be of type {\tt int}.  Otherwise, {\tt inilib} will consider the value
to be of type {\tt std::string}.

\subsubsection{Data Type Conversion}
\label{sec:conversions}

Whenever possible, conversion from one data type to another is handled
by casting to the desired data type.  For example, the conversion
between from an {\tt int} to {\tt bool} is handled by calling {\tt
(bool) value}.  In certain circumstances, a cast will not produce the
desired result, such as casting a {\tt std::string} to a {\tt
double}.  Table~\ref{tbl:conversions} details the conversions used in
{\tt inilib}.

\begin{table}
  \begin{center}
    \begin{tabular}[ht!]{| l | p{.9in} | p{.9in} | p{.9in} | p{.9in} |} 
      \hline
      & \multicolumn{4}{| c |}{Casting from} \\ 
      \multicolumn{1}{|c|}{Casting to} & \multicolumn{1}{c|}{{\tt bool}} & 
      \multicolumn{1}{c|}{{\tt double}} & \multicolumn{1}{c|}{{\tt int}} &
      \multicolumn{1}{c|}{{\tt std::string}} \\ 
      \hline
      
      {\tt bool} & NA & {\tt (bool) value} & {\tt (bool) value} & {\tt false}
      if string is empty, {\tt true} otherwise. \\ 
      \hline
      
      {\tt double} & {\tt (double) value} & NA & {\tt (double) value}
        & {\tt atof(value)} \\ 
      \hline
      
      {\tt int} & {\tt (int) value} & {\tt (int) value} & NA & {\tt
      atoi(value)} \\ 
      \hline
      
      {\tt std::string} & ``true'' if {\tt value} is true, empty string
      otherwise & {\tt sprintf(ret, ``\%.{\sc Precision}lf'', value)} &
      {\tt sprintf(ret, ``\%.{\sc Precision}lf'', value)} & NA \\  
      \hline
    \end{tabular}
  \end{center}
  \caption{Conversion behavior in {\tt inilib}}
  \label{tbl:conversions}
\end{table}

